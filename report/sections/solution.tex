\section{Problem Solution}
\label{chap:solution}

\subsection{Design Assumption}

- an elf is implemented as an iterative process 
- we assume the elves batch size and the total number of reideers in the system to be passed as configuration parameters to the program
- in order to introduce some randomness and some non-determinism we assume that every time an elf build a toy there is a random possibility to end up in a failure (that lead the elf to ask santa for help).
- we assume the elves to have a fixed maximum number of failures (can be set as a configuration parameter). after an elf reaches this bound he is basically killed (his lyfecycle is stopped) even if it is still not christmas.
- in order to increase the non-determinism and to make the concurrency more evident, for each elf and for each reindeer we introduce random waiting times at the beginning and the end of their life cycles


\subsection{Solution Description}

- talk about the code design: 
- introduce the class and BOM diagram
- focus on the elf class
- focus on the reindeer class
- focus on santa class
- show and comment state diagram
- sequence to sum up the system behavior

- focus on critical points:
- what happens when santa leave?
- .....
