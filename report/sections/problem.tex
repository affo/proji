\section{Problem Description}
\label{sec:problem}

\subsection{The Problem}
\label{subsec:problem}
In this section we highlight the problem taken from
\textit{The Little Book of Semaphores}\footnote{\url{http://greenteapress.com/semaphores/}}:

``Santa Claus sleeps in his shop at the North Pole and can only be awakened by
either $(1)$ all nine reindeers being back from their vacation in the South
Pacific, or $(2)$ some of the elves having difficulty making toys; to allow Santa
to get some sleep, the elves can only wake him when three of them have
problems. When three elves are having their problems solved, any other elves
wishing to visit Santa must wait for those elves to return. If Santa wakes up
to find three elves waiting at his shop’s door, along with the last reindeer
having come back from the tropics, Santa has decided that the elves can wait
until after Christmas, because it is more important to get his sleigh ready
(It is assumed that the reindeers do not want to leave the tropics, and
therefore they stay there until the last possible moment). The last reindeer
to arrive must get Santa while the others wait in a warming hut before being
harnessed to the sleigh.

Here are some additional specifications:

\begin{itemize}
\item After the ninth reindeer arrives, Santa must invoke \code{prepareSleigh}, and
then all nine reindeer must invoke \code{getHitched}.
\item After the third elf arrives, Santa must invoke \code{helpElves}. Concurrently,
all three elves should invoke \code{getHelp}.
\item All three elves must invoke \code{getHelp} before any additional elves enter
(increment the elf counter).
\end{itemize}

Santa should run in a loop so he can help many sets of elves. We can assume
that there are exactly $9$ reindeers, but there may be any number of elves.''

\subsection{Specifications and Assumptions}
\label{subsec:specifications}
From the problem described in section \ref{subsec:problem}, we can derive some
more specifications and assumption:

\begin{enumerate}
\item ``\ldots the elves can only wake him when three of them have problems
\ldots'': the elves have to be processed in batches of $3$;

\item ``When three elves are having their problems solved, any other elves
wishing to visit Santa must wait for those elves to return'': when an elf wants
to go to Santa's and Santa is busy, he/she has to wait the elves at Santa's to come
back;

\item ``\ldots it is more important to get his sleigh ready \ldots'': in case $9$
reindeers have come and $3$ elves are at the door, Santa will start preparing
the sleigh;

\item ``After the ninth reindeer arrives, Santa must invoke \code{prepareSleigh}, and
then all nine reindeer must invoke \code{getHitched}'': there is nothing said
about what the elves can do in this case. We assume that elves can still come
while Santa is hitching and they will be treated as specified before;

\item ``After the third elf arrives, Santa must invoke \code{helpElves}.
Concurrently, all three elves should invoke \code{getHelp}'': as before,
nothing is said about reindeers. We assume that reindeers can come while Santa
is helping elves;

\item Every elf that goes to Santa comes back only after being helped.
\end{enumerate}
